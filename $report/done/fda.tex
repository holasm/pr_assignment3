\begin{center}
\section{FDA (Fisher Discriminant Analysis)}
\end{center}

\subsection{Introduction}
\begin{flushleft}
Fishers linear discreminent analysis technique is used to reduce the dimension of the n-dimensional point to the k-dimension where k <= n. FLDA reduces the dimensions in the direction of maximum of between the class scatter and minimum of the withiin class scatter. 
\end{flushleft}

\subsection{Goal}
\begin{flushleft}
    Identify image using FDA (Fisher Discriminant Analysis).
\end{flushleft}

\subsection{Data}
\begin{flushleft}
    Given files were containg extracted features from different images in terms of 36x23 dimention matrix.
\end{flushleft}


\subsection{Experiment}
\begin{flushleft}
  At first we reduced the dimensions of every input feature vector from 23 to 7 (no. of classes - 1) and then we applied K-NN classifier to classify.
  
\begin{figure}[!htb]
\begin{center}
\minipage{0.9\textwidth}
  \includegraphics[width=\linewidth]{6.jpg}
  \caption{ROC cureve}\label{fig:fig_a}
\endminipage\hfill
\end{center}
\end{figure}
\break
\begin{figure}[!htb]
\begin{center}
\minipage{0.9\textwidth}
  \includegraphics[width=\linewidth]{7.JPG}
  \caption{Confusion matrix}\label{fig:fig_a}
\endminipage\hfill
\end{center}
\end{figure}
 \break
\end{flushleft}

\subsection{Observation}
\begin{flushleft}
  1. The efficiency depends on the value of k.
\end{flushleft}

\break