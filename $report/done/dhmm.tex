\begin{center}
\section{DHMM (Discrete Hidden Markov Model)}
\end{center}

\subsection{Introduction}
\begin{flushleft}
  DHMM is one kind of supervised learning model. In this model the transition from one symbol to 
  another is clearly known.
\end{flushleft}

\subsection{Goal}
\begin{flushleft}
    Identify digit sequence using DHMM (Discrete Hidden Markov Model).
\end{flushleft}

\subsection{Data}
\begin{flushleft}
  Given files were containg extracted features for diffenent digits individually. 
\end{flushleft}

\subsection{Observation and Preprocessing of Data}
\begin{flushleft}
  Each file contained mutiple feature vector for one digit. Each row contained 39 components.
\end{flushleft}

\subsection{Experiment}
\begin{flushleft}
  We used MIT hmm toolkit for this experiment. 
  First we applied k-means clustering to get discontinuous clusters.
  Then we assigned multiple cluster to different digits (classes).
  After that we applied dhmm\_em() train data with multiple iteration.
  Next we calculated accuracy of test data classification calculating log probability using dhmm\_logprob().

  For 15 states and 50 iteration for dhmm\_em() the accuracy was around \textbf{\textit{82\%}}. \break
\end{flushleft}

\subsection{Observation}
\begin{flushleft}
  1. The accuracy increased with the increase of number of states and iteration count.
\end{flushleft}
