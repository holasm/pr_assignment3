\begin{center}
\section{DTW}
\end{center}

\subsection{Introduction}
\begin{flushleft}
  DTW(Dynamic Time Warping) is used in time series analysis. It can be used to measure similarity between varying length sequence.
  DTW can be used to calculate similarity between time series sequence.  
\end{flushleft}

\subsection{Goal}
\begin{flushleft}
    Identify city names from speech using DTW (Dynamic Time Warping).
\end{flushleft}

\subsection{Data}
\begin{flushleft}
  Given .wav file and .mfcc files corresponding to differnent cities of south India.
  The .mfcc files contains the features extracted from .wav files.
\end{flushleft}


\subsection{Observation and Preprocessing of Data}
\begin{flushleft}

  It was observed that for many cities there was not enoung data i.e. enough .mfcc file was not there.
  So we reduced the data set by ignoring .mfcc files corresponding to cities having less than 20 .mfcc files.\break
  
  Then the reduced data was divided into training set by taking \textbf{\textit{25\%}} of .mfcc files for each city.
  The remaining \textbf{\textit{75\%}} data (.mfcc files) was used for training.

  \break
  \textbf{Before} data reduction \break
  Total number of given .mfcc file: 2737 \break
  
 \textbf{After} data reduction \break
 Total number of test .mfcc file: 216 \break
 Total number of train .mfcc file: 539 \break

  \textbf{Note}: When .wav files are not given we can generate .mfcc from .wav files using matlab.
\end{flushleft}

\subsection{Experiment}
\begin{flushleft}
  \textbf{Approach 1}: \break
    We have tried two approaches to classify the test cities.
  Our approach was to calculate dtw distance between one test (.mfcc) file and all train (.mfcc) file.
  From the train (.mfcc) file, giving minimum dtw distance is taken as classified city. All the coding is done
  in c++.
  
  With this approach we got around \textbf{\textit{20\%}} efficiency.

  The approach is shown pictorially below.

\begin{figure}[!htb]
\begin{center}
\minipage{0.8\textwidth}
  \includegraphics[width=\linewidth]{1.png}
  \caption{Approach 1}\label{fig:fig_a}
\endminipage\hfill
\end{center}
\end{figure}

\break

  \textbf{Approach 2}: \break
    The first approach was not giving resonable efficiency. So we tried averaging the dtw distance for all train (.mfcc) files having same city name. This time we tried python.
    
    With this approach we got around \textbf{\textit{35\%}} efficiency.

    The approach is shown pictorially below
\begin{figure}[!htb]
\begin{center}
\minipage{0.8\textwidth}
  \includegraphics[width=\linewidth]{2.png}
  \caption{Approach 2}\label{fig:fig_a}
\endminipage\hfill
\end{center}
\end{figure}
\end{flushleft}

\subsection{Inferances}
\begin{flushleft}
  1. We need more data set to get reasonable classification.
  2. The dtw algorithm is somewhat similar to LCS(Least Common Subsequence) with little modification.
  3. Classification using dtw is extremely computation intensive.
\end{flushleft}

\subsection{Links}
\begin{flushleft}
  All the codes are available at the following
  \color{red} 
    \href{https://github.com/holasm/pr_assignment3/tree/master/dtw}{link}.
  \end 
  4045b2ac7aadbc011301ccf731a1fb5094e8000b
\end{flushleft}
