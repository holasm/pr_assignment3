\begin{center}
\section{HMM - HANDWRITTEN data}
\end{center}

\subsection{Introduction}
\begin{flushleft}
  HMM is a statistical model where we use supervised learning to create the model.
  The model contains hidden states. We attach state transition probability to each state.
  The states can emit symbols with certain probabilities. \break
  The state transition probabilities and 
  symbol emission probabilities are calculated using Baum Welch algorithm.
  The classification of a particular sequence is done using Vitterb algorithm on the constructed HMM.
\end{flushleft}

\subsection{Goal}
\begin{flushleft}
    Identify character symbols using HMM(Hidden Markov Model) using HTK tool kit.
\end{flushleft}

\subsection{Data}
\begin{flushleft}
    Given .ldf files corresponding to diffenent character symbols.
  The .ldf files were created by extracting feature from pictures of character symbols.
  Each files contain x, y coordinate for diffenent points on the symbol picture.
  All the train and test data was not already separated.
\end{flushleft}


\subsection{Observation and Preprocessing of Data}
\begin{flushleft}
    
    All given .ldf files were containing data for multple picture of same character symbols.
  We have created multiple .mfcc files from the x, y coordinated for one feature corresponding 
  to one character symbol picture.
  Then we separated \textbf{\textit{25\%}} of .mfcc files for each character symbol and taken as test data.
  The remaining \textbf{\textit{75\%}} .mfcc files were concisered for training.

  We were about to use HTK toolkit to do the experiment.
  The following stepas were taken\break
  \textbf{1}. To use the HTK tool kit we need to convert the .mfcc files to .htk format. To do this we have used matlab script file.\break
  \textbf{2}. To use the matlab script we need to truncate the first line of the provided .mfcc files.\break
  \textbf{3}. To use the matlab script we also need a filelist file with all .mfcc file paths.\break

  Note: When .wav files are not given we can generate .mfcc from .wav files using matlab.
\end{flushleft}

\subsection{Experiment}
\begin{flushleft}
  We have HTK toolkit for the entire experiment.
  
  The steps taken as follows. \break
  
  1. Create grammer file \break
  \begin{lstlisting}[language=bash]
    $ HParse ... (i/p: grammer, o/p: wordnet)
  \end{lstlisting}
  2. Create {wlist} file \break
  
  3. Create {lexicon} file \break
  \begin{lstlisting}[language=bash]
    $ HDMan ... (i/p: wlist, lexicon, model0, o/p: dict, dlog)
 \end{lstlisting}
  4. Create {symbol.mlf} file. The all.mlf file contains specific symbol label. \break
  
  5. Create {symbol.scp} file. This is the training scp file. The symbol.scp file contains specific symbol .mfcc paths. \break
  
  6. Create {proto} file \break
  \begin{lstlisting}[language=bash]
    $ HInit ... (i/p: all.scp, proto, o/p: newproto) \break 
  \end{lstlisting}
    \begin{lstlisting}[language=bash]
    $ HRest ... (i/p: macros, hmmdefs, o/p: newproto)  \break 
  \end{lstlisting}
    Repeat \$ HRest for 4 or more times using previously generated proto for each character symbol.
  
  
  \begin{lstlisting}[language=bash]
    $ HERest ... (i/p: macros, hmmdefs, o/p: newproto)
    \end{lstlisting}
    (repeat 4 or more times using previously generated proto)
    \break
    
    \begin{lstlisting}[language=bash]
    $ HHEd ... (i/p: macros, hmmdefs, o/p: newHmmdefs.mlf)
    \end{lstlisting}

    
    \begin{lstlisting}[language=bash]
        $ HResults ... (i/p: allTest.mlf, o/p: result.mlf, confusiton matrix)
    \end{lstlisting}
    
    In the above experiment we have create created separate hmmdef for different character symbols.
  Then after applying HRest ...  we have copied all generated proto (lastly) file in hmmdefs file. 
  We have created the test.scp by kepping all test (.htk) file paths (for all character symbols).
  We have also created the test.mlf by kepping all labels for test .mfcc files (for all character symbols).
  
  In the above experiment we have tried using \break
     5 states and 1 mixture for each digit with efficiency \textbf{\textit{50\%}}. \break
    8 states and 1 mixture for each digit with efficiency \textbf{\textit{60\%}}. \break
    12 states and 5 mixture for each digit with efficiency \textbf{\textit{85\%}}. \break
 \break
\end{flushleft}

\subsection{Inferances}
\begin{flushleft}
  1. It was observed that with the increase of state number and mixture for each digit the effenciy increases.
\end{flushleft}

\subsection{Links}
\begin{flushleft}
  All the codes are available at the following
  \color{red} 
    \href{https://github.com/holasm/pr_assignment3/tree/master/hmm/handwritten}{link}.
  \end 
  4045b2ac7aadbc011301ccf731a1fb5094e8000b
\end{flushleft}