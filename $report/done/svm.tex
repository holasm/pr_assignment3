\begin{center}
\section{SVM (Support Vector Machine)}
\end{center}

\subsection{Introduction}
\begin{flushleft}
  SVM (Support Vector Machine) is mainly used in linear classification. It is a supervised learning model.
  In SVM the training data is marked as belonging to one of two categories separeated by line. This way we 
  classify new test data by observing which side th test data falls. \break
  We can extend this idea for classification of among multiple class. 
\end{flushleft}

\subsection{Goal}
\begin{flushleft}
    Identify image using SVM (Support Vector Machine). We had to use libsvm library.
\end{flushleft}

\subsection{Data}
\begin{flushleft}
    Given files were containg extracted features from different images in terms of 36x23 dimention matrix.
\end{flushleft}

\subsection{Observation and Preprocessing of Data}
\begin{flushleft}
  We have taken 25\% of the given data from each class and used it as test data.
  The remaining 75\% was used as training data.
\end{flushleft}

\subsection{Experiment}
\begin{flushleft}
  We used two library function provided by libsvm toolkit.\break
  
  1. svmtrain()\break
  2. svmpredict()\break

  For svmtrain() function the first argument was a matrix of feature vectors created 
  by vertically concatenating all training image feature matrix.
  The second argument was a vector indicating the class label for row of the first argument matrix.
  The output of the svmtrain() is the train\_model. \break

  For svmpredict() function the first argument was a matrix of feature vectors created 
  by vertically concatenating all test image feature matrix.
  The second argument was a vector indicating the class label for row of the first argument matrix.
  The second argument was the train\_model we got as output from svmtrain().
  
\begin{figure}[!htb]
\begin{center}
\minipage{0.9\textwidth}
  \includegraphics[width=\linewidth]{3.png}
  \caption{ROC cureve}\label{fig:fig_a}
\endminipage\hfill
\end{center}
\end{figure}
  
\end{flushleft}
\break