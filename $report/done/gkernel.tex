\begin{center}
\section{Parzen Window - using Gausian Kernel}
\end{center}

\subsection{Introduction}
\begin{flushleft}
 Parzen Windiow is a non-parametric way of classifying data. In Gausian Kernel we calculate probability 
 of all training points and the test point is classified as in the class of the point with highest probaility.
\end{flushleft}

\subsection{Goal}
\begin{flushleft}
    Identify image using Parzen Window (using Gausian Kernel).
\end{flushleft}

\subsection{Data}
\begin{flushleft}
    Given files were containg extracted features from different images in terms of 36x23 dimention matrix.
\end{flushleft}


\subsection{Observation and Preprocessing of Data}
\begin{flushleft}
  We created one vector for each image data of dimention 1x828 (23x36).
\end{flushleft}

\subsection{Experiment}
\begin{flushleft}
  The efficiency was around \textbf{\textit{60\%}}
\end{flushleft}

\subsection{Observation}
\begin{flushleft}
  1. The efficiency depends on the radius of the hypersphere.
  There is a optimum value of radius of the hypersphere for which the efficiency will be maximum.
\end{flushleft}
