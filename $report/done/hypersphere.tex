\begin{center}
\section{Parzen Window - Hypersphere as A Region}
\end{center}

\subsection{Introduction}
\begin{flushleft}
 Parzen Windiow is a non-parametric way of classifying data. In this method we take a volume in
  space around the test feature vector. For Hypersphere as a region we take hypersphere volume
  and calculate number of points from different classes inside the sphere.
  The class in which maximum points are lying is the resultant class of input feature vector.
\end{flushleft}

\subsection{Goal}
\begin{flushleft}
    Identify image using Parzen Window (Taking Hypersphere as A Region).
\end{flushleft}

\subsection{Data}
\begin{flushleft}
    Given files were containg extracted features from different images in terms of 36x23 dimention matrix.
\end{flushleft}


\subsection{Observation and Preprocessing of Data}
\begin{flushleft}
  We created one vector for each image data of dimention 1x828 (23x36).
\end{flushleft}

\subsection{Experiment}
\begin{flushleft}
    We have tried different volume of hyperspheres yielding varying effeciencies.

    For h = 25 the efficiency was \textbf{\textit{42\%}}
    For h = 20 the efficiency was \textbf{\textit{48\%}}
    For h = 15 the efficiency was \textbf{\textit{55\%}}
    For h = 10 the efficiency was \textbf{\textit{50\%}}
\end{flushleft}

\subsection{Observation}
\begin{flushleft}
  1. The efficiency depends on the radius of the hypersphere.
  There is a optimum value of radius of the hypersphere for which the efficiency will be maximum.
\end{flushleft}
